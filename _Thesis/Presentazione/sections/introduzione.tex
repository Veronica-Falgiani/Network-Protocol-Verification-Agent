\section{Introduzione}

%=======================================================================

\begin{frame}{Vulnerability Scanner}

I \textbf{Vulnerability Scanner} sono programmi che interagiscono con i dispositivi in rete inviando pacchetti e analizzando le risposte per individuare:
\begin{itemize}
    \item \textbf{Vulnerabilità}
    \item \textbf{Configurazioni errate}
    \item \textbf{Versioni obsolete}
\end{itemize}
\begin{columns} 
    \begin{column}{0.3\textwidth}
    \begin{figure}
    \centering
    \includegraphics[width=2.6cm]{assets/mio/nessus.png}
    \end{figure}
    \end{column}
    \begin{column}{0.3\textwidth}
    \begin{figure}
    \centering
    \includegraphics[width=2.6cm]{assets/mio/nmap.png}
    \end{figure}
    \end{column}
    \begin{column}{0.3\textwidth}
    \begin{figure}
    \centering
    \includegraphics[width=2.6cm]{assets/mio/metasploit.jpg}
    \end{figure}
    \end{column}
\end{columns}
\end{frame}

%=======================================================================

\begin{frame}{Criticità dello Stato dell'Arte}

I vulnerability scanner citati precedentemente soffrono di alcune problematiche:
\begin{itemize}
    \item \textbf{Complessità}
    \item \textbf{Dimensioni eccessive dell'applicativo}
    \item \textbf{Curva di apprendimento ripida}
\end{itemize}


\end{frame}

%=======================================================================

\begin{frame}{Soluzione Proposta}
    L'\textbf{obiettivo} di questo elaborato consiste nello sviluppo di un vulnerability scanner che abbia come caratteristiche principali:
    \vspace{0.6cm}
    \begin{columns} 
    \begin{column}{0.3\textwidth}
    \begin{block}{Modularità}
    \end{block}
    \end{column}
    \begin{column}{0.3\textwidth}
    \begin{block}{Leggerezza}
    \end{block}
    \end{column}
    \begin{column}{0.3\textwidth}
    \begin{block}{Semplicità d'uso}
    \end{block}
    \end{column}
    \end{columns}
\end{frame}